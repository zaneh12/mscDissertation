\addstarredchapter{Abstract}
\chaptermark{Abstract}

\chapter*{Abstract}
The United Kingdom alone has around 8.5 Trillion Pounds worth of assets under management, and fund managers must make difficult asset allocation decisions. A British company, Irithmics, built an agent based machine learning model to predict, based upon the prior fund management behavior, the probability of the entire group of funds to short a given stock on a given day. This data was graciously donated for my M.Sc. Dissertation by company founder and St Andrews Alumni Grant Fuller. This paper seeks to determine whether the "Wisdom of the Crowds" holds any merit when trying to forecast the volatility of stock market returns. It specifically looks at four constituents of the FTSE 100 (Lloyd's, Tesco, Rolls Royce, Vodafone) during one of the largest times of uncertainty for the stock market known to date, the year of 2020. The research focuses determining whether the exogenous data provided by Irithmics can increase the 1-day ahead forecast accuracy of an optimized univariate empirical conditional volatility model. The compelling aspect of the research question is the exogenous data may not necessarily be an accurate prediction of the fund managers behavior, yet it still may be an advantageous addition to the model.   