\chapter{Literature Review}


\section{Volatility Modeling of Stocks from Selected Sectors of the Indian Economy Using GARCH}
\subsection{Overview}
\cite{9544977} is a recent paper presented at the 2021 Asian Conference on Innovation in Technology (ASIANCON). This paper visits the idea of volatility clustering applied to the Indian Economy and therefore on Indian stocks on the NSE. The paper focuses on applying only the concept of \textbf{GARCH} or Generalised AutoRegressive Conditional Heteroscedasticity models to forecast volatility of returns rather than incorporating any exogenous variables or data.
\subsection{Methodology}
As the models the researchers considered were quite straightforward and easy to implement in Python, they took a very structured approach to the problem by beginning with the most simple \textbf{GARCH(1,1)} model, and then added complexity. This meant beginning with a constant mean model and normally distributed residuals, followed by skewed -t distributed resiudals. They then fit an Autoregressive Moving Average mean model residuals into the model and took the minimum AIC. Then they fit Asymmetric volatility models on the return series, this for example is the \textbf{GJR-GARCH} and \textbf{EGARCH} models which asses the impact of a negative shock as more impactful than a positive one. 
\subsection{Results}
The results were completed by testing the best model by minimum AIC with some out of sample data that the model has not seen. They used an expanding window and fixed window forecast and backtested the \textbf{EGARCH} Model. This was evaluated by a lot of different statistics on the Auto and Banking Sector and \textbf{EGARCH} in this scenario was best chosen.
\section{An Empirical Study of Hang Seng Index Based on GARCH Model}
\subsection{Overview}
\cite{LR2} is a recent paper presented at the 2020 second international conference on Economic Management and Model Engineering (ICEMME). The aim was to give an overview of optimal model fitting on the Hang Sang Index. The HSK is 33 stocks, akin to FTSE100, that look to give an overview of the economic and financial activity and health of the honk kong markets. The power of this index is it includes mainland china based companies in addition to Hong Kong. As discussed, the paper admits that the volatility of stock market returns are time-variant and therefore traditional measurements of standard deviation of the whole sample are useless. The goal of this paper was to evaulate what variant of a \textbf{GARCH(p,q)} model would fit best to the data
\subsection{Methodology}
Contrary to R's rugarch, or Python's arch library the researchers used Eviews10 software accompanied by the stock market data provided by the Flush Database. The returns calculated were log or continuous returns ($r_t = ln(P_t)-ln(P_{t-1})$). They did exploratory analysis on the data and determined it was stationary, and it was correlated enough to continue their analysis. To select the model they fit \textbf{GARCH(1,1)}, \textbf{GARCH(2,1)}, \textbf{GARCH(1,2)}, \textbf{GARCH(2,2)} and picked the best model from the Akaike Information Criteria or AIC. Using minimum AIC they chose the \textbf{GARCH(1,2)}. 

\subsection{Results}
The main takeaways the researchers had from this analysis were that the continuous returns (log change in daily prices) of the Hang Seng Index did have volatility clustering. The returns also had from the stylized facts fatter tails and sharper peaks in the distribution (similar to $\mathbf{Figure~1.2}$). They also determined there were no unit roots in the returns sequence, therefore inferred signs for stability, while through analysis of autocorrelation there was no monthly correlation. The drawbacks were the lack of sample size, but found as usual that the \textbf{GARCH} model's Variance predictions were very adequate for stock market returns. 

\section{A GARCH approach to model short-term interest rates: Evidence from Spanish economy}
\subsection{Overview}
\cite{https://doi-org.ezproxy.st-andrews.ac.uk/10.1002/ijfe.2234} was published in September 2020 in the International Journal of Finance and Economics. Instead of talking about stock market returns, this paper gives a derivative approach, modelling short term interest rates of three year Spanish government issued bonds. Again, this process is completed because the researcher believes that \textbf{GARCH} models provide: "a valuable alternative against econometric specifications that imply a homoscedastic error term" \cite{https://doi-org.ezproxy.st-andrews.ac.uk/10.1002/ijfe.2234}. The data is taken from January 1995 to December 2000 and is useful as there will be exogenous volatility caused by what they say is the European Central Bank assuming monetary sovereignty. 
\subsection{Methodology}
The paper takes into account interest rate theory and the exogenous factors that affect the rate that money is returned as a profit for the buyer of the bond, there is expectations, liquidity preference, institutional approach, habitat hypothesis, market segmentation hypotheses \cite{https://doi-org.ezproxy.st-andrews.ac.uk/10.1002/ijfe.2234}. They then applied a mathematical eiquation to the problem through the fisher information equation $$i_R = i_N -\pi$$ This effectively is the real interest rate is equal to the nominal interest rate minus the expected inflation rate. With this information they fit a \textbf{GARCH} model with the interest rates as the input hoping to obtain the conditional volatility of the time series
\subsection{Results}
The researchers found that this method of estimating the interests rates when breaking the traditional assumption of homoscedastic variance, instead with the heteroscedastic that \textbf{GARCH} employs was much more efficient and accurate. This model was more flexible as well, but a sub-specification of the \textbf{GARCH} model worked better than fitting through the traditional equation. The fisher equation was effective in its estimation of the government debt and they believe this can be expanded into other fields of finance and econometrics as well. \cite{https://doi-org.ezproxy.st-andrews.ac.uk/10.1002/ijfe.2234}